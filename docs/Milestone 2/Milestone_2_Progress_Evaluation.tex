\documentclass[12pt]{article}
\newcommand{\namesigdate}[2][5cm]{%
  \begin{tabular}{@{}p{#1}@{}}
    #2 \\[2\normalbaselineskip] \hrule \\[0pt]
    {\small \textit{Signature}} \\ [2\normalbaselineskip] \hrule \\[0pt]
    {\small \textit{Date}}
  \end{tabular}
}
\usepackage{adjustbox}
\usepackage{color} 
\usepackage{tabularx}
\usepackage{authblk}
\usepackage{tabto}
\usepackage{tcolorbox}
\usepackage{setspace}
\usepackage{listings}
\lstset { language=bash,firstnumber=1,numbers=left,numbersep=-10pt,basicstyle=\ttfamily,
  showstringspaces=false,
  commentstyle=\color{red},
  keywordstyle=\color{blue}}
\usepackage[document]{ragged2e}
\usepackage[left=1in,right=1in,top=1in,bottom=1in]{geometry}
\usepackage[pdftex,pdfpagelabels,bookmarks,hyperindex,hyperfigures]{hyperref}
\usepackage{graphicx}
\begin{document}
	\title{\textbf{Milestone 2 Progress Evaluation} \\ \hfill \break
	Academic Behavior Reccomendation System}
	\author{Shreyas Ugemuge\      \texttt{sugemuge2014@my.fit.edu}
  \and
  Yaqeen AlKathiri\      \texttt{yalkathiri2013@my.fit.edu}
  \and
	Mohammed AlHabsi\      \texttt{malhabsi2013@my.fit.edu}
  \and
  Shiru Hou\      \texttt{shou2015@my.fit.edu}
  \and
  Faculty Sponsor: Dr. Phillip Chan\      \texttt{pkc@cs.fit.edu}}
	\maketitle
	\pagebreak
	\singlespacing
	\tableofcontents
	\pagebreak
	\section{Progress of current Milestone}
	\begin{adjustbox}{totalheight=\textheight-2\baselineskip}
	\begin{tabularx}{\textwidth}{|X|c|c|c|c|c|X|}
	\hline
		\textbf{Task} & \textbf{Completed} &\textbf{Shreyas} & \textbf{Yaqeen} & \textbf{Shiru} & \textbf{Mohammed} & \textbf{Remarks}  \\
		\hline
		Finalize conceptualizing 8 behaviors & 100\% & 28\% & 35\% & 22\% & 15\% & 14 Behaviours were identified \\ \hline
		Finalize packages, languages and tools & 90\% & 80\% & 20\% & - & - & Still need to examine the possibility of WEKA \\ \hline
		Implement and test framework to get input from syllabus & 50\% & - & - & 50\% & 50\% & More details from the syllabus need to be input. More exception handling\\ \hline
		Select data mining techniques for each behavior & 80\% & 25\% & 25\% & 25\% & 25\% & Some behaviors will have their method determined based on implementation\\ \hline
		Provide report explaining behaviours and corresponding data mining methods & 100\% &25\% & 25\% & 25\% & 25\% & N/A\\ \hline
		Implement and extract data for 2 behaviors & 100\% &70\% & 30\% & - & - & 3 behaviours implemented and extracted\\ \hline
		Begin Preparing dataset & 80\% & 70\% & 30\% & - & - & Need to use Regex/pyEnchant to get rid of non human-readable strings \\  \hline
		Update Requirements Documents & 100\% & - & - & 50\% & 50\% & N/A \\ \hline
		Progress Evaluation & 100\% & 40\% & 20\% & 20\% & 20\% & N/A \\ \hline 
	\end{tabularx}
	\end{adjustbox}
	\subsection{Discussion of each task}
		\subsubsection{Finalize Conceptualizing 8 behaviors}
		This task included examining the data to find behaviors that would aid in correlation to the grades. The team conceptualized 14 behaviors as opposed to the planned 8:
		\begin{enumerate}
			\item Number of days for each student with 0 activity\footnote{Implemented}
			\item Total number of Logins throughout the semester\footnote{Implemented}
			\item Average activities for each login \footnote{Implemented}
			\item Time between a due date and the first time relevant course material is accessed
			\item Average weekly Logged in time
			\item Number of assignments submitted on time
			\item Average review time before assignments
			\item Average review time before tests
			\item Number of optional assignments done
			\item Total time accessing study guides
			\item Total time accessing podcasts
			\item Frequency of accessing study guides
			\item Frequency of Quiz reviews
			\item Average time taken for quizzes
		\end{enumerate}
		
		\subsubsection{Finalize packages, languages and tools}
		The primary programming language was agreed upon to be \textbf{Python}, due to the high availability and quality
		of available open source data mining tools and libraries. Python also provides great API and framework to handle different file types and to present data
		The python libraries decided and implemented are:
		
		\begin{itemize}
			\item numpy - used to handle numbers and distributions, very widely and generically used library
			\item scipy - used to more advanced distributions, moments and kurtosis.
			\item scikit-learn - open source data mining/ machine learning library
			\item statsmodels 
			\item pandas - makes working with tabular data like csv files very simple
			\item matplotlib - presentation
			\item pyEnchant  - To distinguish plain english from gibberish and encoded strings
			\item sys - command line option parsing
		\end{itemize}
		Weka is being examined as a prospective tool
		
		\subsubsection{Implement and test framework to get input from syllabus}
		Simple CLI was built using python. To get all the surface data. This will be further detailed with having more parameters and better exception handling as well as data validation.
		\subsubsection{Select data mining techniques for each behavior}
		\url{https://github.com/shreyasugemuge/Senior-Design-Project/tree/master/docs/Milestone\%202/Behaviors} comprises of four files with each group members research and analysis. 3 have been implemented, another layer of refinement is expected. Multiple Data mining methods for a single behavior may be used in order to find the best metric.
		\subsubsection{Clean and prepare data}
		The repository has been set up in a way that allows for the data to be on the local computer but ignored while being pushed. This enables a local working directory compatible to the scripts and programs written. The program CleanDataset.py serves to create a more program friendly log file. This will be refined if required. One thing that will be addded is filtering for encoded strings so the sample size of activities can be limited and hence quantified.  \\ \hfil \break
		Certain Issues were identified and dealt with in this part, for instance the column containing quiz information also had duplications of all the object information. These were removed. The column titles were changed along with filenames for easier access. \\ \hfil \break
		A supervised Item based collaborative filtering would suggest that 0 as an activity is actually an outlier for the data And sessions that login and abandon the session should be treated as erronous data. This information was preserved however in order to treat them as a metric for behavior
		\subsubsection{Implement and extract data for 2 behaviors}
		3 behaviors were implemented as mentioned in 1.1.1. This subsection will explain the source directory structure and the programs.
		\begin{itemize}
			\item src/Behaviors
			\begin{itemize}
				\item run.sh run the program in correct order and provides a verbose output while doing so
				\item Behavior1.sh runs the beh\_1.py to extract 3 behaviours
				\item CleanDataset.sh run CleanDataset.py to clean all log files
			\end{itemize}
		\end{itemize}
		\subsubsection{Update documents}
		Requirements document was updated, this will continue for the next milestone. The progress evaluation was drafted.
		
	\subsection{Discussion of team member contribution}
	\subsubsection{Shreyas}
		Conceptualized 4 Behaviors. Implemented 3 Behaviors. Finalized the python end of the program, including libraries. Wrote python script to prepare dataset. Updated website. Explored and dismiss possibility of unsupervised learning models. Explored possibility of of user based collaborative filtering using the given survey data, to further classify results. Update documents.
	\subsubsection{Shiru}
	Conceptualized 2 behaviors, provided a report. Contributed to making the program to get syllabus information from the user. Contributed in updating the requirements document as well as progress evaluation.
	\subsubsection{Yaqeen}
	Conceptualized 5 behaviors along with data mining technique. Implemented 1 behaviors using WEKA, and prepared the dataset. Provided a report. Contributed in updating progress evaluation.
	\subsubsection{Mohammed}
	Conceptualized 2 behaviors along with data mining techniques, provided a report. Contributed to making the program to get syllabus information from the user. Contributed in updating the requirements document as well as progress evaluation
	\section{Plan for the next milestone}
	\begin{tabularx}{\linewidth}{|X|X|X|X|X|}
	\hline
	\textbf{Task} & \textbf{Shreyas} & \textbf{Yaqeen} & \textbf{Shiru} & \textbf{Mohammed} \\ \hline
	Finish extracting information using the behaviors (16 more behaviors to be implemented) & 25\% & 25\% & 25\% & 25\% \\ \hline
	Begin Correletion of behavior information with grades & 25\% & 25\% & 25\% & 25\% \\ \hline
	Present information as plots, graphs and charts & 25\% & 25\% & 25\% & 25\% \\ \hline
	Design and begin implementing GUI. Must finish The design and UI side. & 25\% & 25\% & 25\% & 25\% \\ \hline
	\end{tabularx}

	\subsection{Discussion of each task}
	\subsubsection{Finish extracting information using the behaviors}
	The plan is to conceptualize 16 more behaviors for a total of 30. All of the implementations and information will not be used for the correlation. Decision trees to find the least entropy and lest chances of overfitting will qualify data, which is why having an abundance of behaviour will help. 
	\subsubsection{Begin Correletion of behavior information with grades}
	In this part correlations between grades and behaviors will start being computed. Thresholds for confidence intervals will be determined.
	\subsubsection{Present information}
	matplotlib and statsmodels will help generate a good pictorial representation of the data and information we have. The behavior analysis will yield abstract human comprehensible reports for each student and for the class as a whole. 
	\subsubsection{Design and Implement GUI}
	The Teacher end and Student end must be designed and clearly distinguished, with proper channels of information management to ensure product security. GUI will be mainly buttons, fields and check boxes for an intuitive yet easy communication.

	\pagebreak
	\section{Sponsor feedback on each task for current milestone}
	\pagebreak
	\section{Sponsor Evaluation}
	Sponsor: Please detach this page and return to Dr. Shoaff \\ \hfill \break 
	Score (0-10) for each member: circle a score (or circle two adjacent scores for .25 or write down a real number between 0 and 10) \\ \hfill \break
	\begin{tabularx}{\textwidth}{|X|c|c|c|c|c|c|c|c|c|c|c|c|c|c|c|}
	\hline
	Shreyas Ugemuge & 0 & 1 &  2 & 3 & 4 & 5 & 6 & 6.5 & 7 & 7.5 & 8 & 8.5 & 9 & 9.5 & 10 \\ \hline
	Yaqeen AlKathiri & 0 & 1 &  2 & 3 & 4 & 5 & 6 & 6.5 & 7 & 7.5 & 8 & 8.5 & 9 & 9.5 & 10 \\ \hline
	Mohammed AlHabsi & 0 & 1 &  2 & 3 & 4 & 5 & 6 & 6.5 & 7 & 7.5 & 8 & 8.5 & 9 & 9.5 & 10 \\ \hline
	Shiru Hou & 0 & 1 &  2 & 3 & 4 & 5 & 6 & 6.5 & 7 & 7.5 & 8 & 8.5 & 9 & 9.5 & 10 \\ 
	\hline 
	\end{tabularx}
	\hfil \break
	\hfil \break
	\namesigdate{Faculty Sponsor}
	\end{document}
