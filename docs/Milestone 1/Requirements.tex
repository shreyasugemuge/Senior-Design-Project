%Shreyas ugemuge, sr design requirements source
\documentclass[12pt]{article}

%%Packages
\usepackage{color}
\usepackage{setspace}
\usepackage{tabularx}
\usepackage{authblk}
\usepackage[document]{ragged2e}
\usepackage[left=1in,right=1in,top=1in,bottom=1in]{geometry}
\usepackage{graphicx}
\usepackage[pdftex,pdfpagelabels,bookmarks,hyperindex,hyperfigures]{hyperref}
%%%%%%%

%%Title and author
\title{\textbf{Requirements Document} \\ \hfill \break
	Student Performance Analysis}
	
\author{Shreyas Ugemuge\      \texttt{sugemuge2014@my.fit.edu}
  \and
  Yaqeen AlKathiri\      \texttt{yalkathiri2013@my.fit.edu}
  \and
	Mohammed AlHabsi\      \texttt{malhabsi2013@my.fit.edu}
  \and
  Shiru Hou\      \texttt{shou2015@my.fit.edu}
  \and
  Faculty Sponsor: Dr. Phillip Chan\      \texttt{pkc@cs.fit.edu}}
  
  \date{\today \\ v1.0}
%%%%%%%%%%

\begin{document}
	\singlespacing
	\maketitle \pagebreak \tableofcontents \pagebreak
	\section{Introduction}
	\subsection{Purpose} \label{purpose}
	The purpose of this document is to highlight the requirements in terms of features and functions that are expected from the final deliverable that will be a recommender system. 
	\subsection{Scope} \label{scope}
	The final product will be an Academic Behaviour Recommendation System. The software will use the information gathered from click log files, syllabi and student grades to recommend for a certain student entry in that particular class the behaviours that were generally observed in that class which led to a good grade but were missing from that particular student's behaviour. The information about the behaviours is computed using a log file containing data about websites visited by the student during the course of the semester the class was taken in and does not consider what the student does beyond the web browser.
	\subsection{Definitions, Acronyms and Abbreviations} \label{defs}
	\begin{itemize}
		\item ABRS - Academic Behaviour Recommendation System
		\item Behaviours - when used in the context of the data means the model in which the data is correlated to resemble a certain behaviour in real life. for instance, a student accessing an assignment link only 1 day before it was due may be a behaviour that may be interesting while correlating to grades. 
		\item Log files - Excel files containing raw data i.e. 'clicks' from student over the course of the semester
		\item Syllabus - Document entailing details about a certain course. This includes test dates, assignment due dates etc.
		\item Grades - Lettered grade ranging from A - F where A is the best grade.
	\end{itemize}
	\subsection{References} \label{refs}
		\begin{enumerate}
		\item IEEE SRS standards guide -  \url{http://www.math.uaa.alaska.edu/~afkjm/cs401/IEEE830.pdf}
		\item IEEE SRS \LaTeX Guide - \url{https://github.com/jpeisenbarth/SRS-Tex/blob/master/srs.tex}
		\item Requirements document specifications - \url{http://cs.fit.edu/~pkc/classes/seniorProjects/document.html}
	\end{enumerate}
	\subsection{Overview} \label{overview}
	The entirety of this document is formatted and authored using the IEEE standards guide as well as the Requirements specified by the Instructor (See \ref{refs}). The remainder of the document contains a detailed description of the project (See \ref{desc}) and requirements (See \ref{reqs}) categorized by functional and non-functional. 
	\section{Overall Description} \label{desc}
	\subsection{Product Perspective}
	\subsection{Product functions}
	\subsection{User characteristics}
	\subsection{Constraints}
	\subsection{Assumptions and dependencies}
	\section{Specific Requirements} \label{reqs}
	
	\end{document}