\documentclass[12pt]{article}
\newcommand{\namesigdate}[2][5cm]{%
  \begin{tabular}{@{}p{#1}@{}}
    #2 \\[2\normalbaselineskip] \hrule \\[0pt]
    {\small \textit{Signature}} \\ [2\normalbaselineskip] \hrule \\[0pt]
    {\small \textit{Date}}
  \end{tabular}
}
\usepackage{color}
\usepackage{tabularx}
\usepackage{authblk}
\usepackage{tabto}
\usepackage{tcolorbox}
\usepackage{setspace}
\usepackage{listings}
\lstset { language=bash,firstnumber=1,numbers=left,numbersep=-10pt,basicstyle=\ttfamily,
  showstringspaces=false,
  commentstyle=\color{red},
  keywordstyle=\color{blue}}
\usepackage[document]{ragged2e}
\usepackage[left=1in,right=1in,top=1in,bottom=1in]{geometry}
\usepackage[pdftex,pdfpagelabels,bookmarks,hyperindex,hyperfigures]{hyperref}
\usepackage{graphicx}
\begin{document}
	\title{\textbf{Milestone 1 Progress Evaluation} \\ \hfill \break
	Student Performance Analysis}
	\author{Shreyas Ugemuge\      \texttt{sugemuge2014@my.fit.edu}
  \and
  Yaqeen AlKathiri\      \texttt{yalkathiri2013@my.fit.edu}
  \and
	Mohammed AlHabsi\      \texttt{malhabsi2013@my.fit.edu}
  \and
  Shiru Hou\      \texttt{shou2015@my.fit.edu}
  \and
  Faculty Sponsor: Dr. Phillip Chan\      \texttt{pkc@cs.fit.edu}}
	\maketitle
	\pagebreak
	\singlespacing
	\tableofcontents
	\pagebreak
	\section{Progress of current Milestone}
	\begin{tabularx}{\textwidth}{|X|c|c|c|c|c|X|}
	\hline
	\textbf{Task} & \textbf{Completion \%} & \textbf{Mohammed} & \textbf{Shreyas} & \textbf{Yaqeen} & \textbf{Shiru} & \textbf{To do} \\ \hline
	1. Select Collaboration Tools & 100 & 20 & 40 & 20 & 20 & n/a \\ \hline
	2. Investigate packages, languages and tools & 100 & 25 & 25 & 25 & 25 & n/a \\ \hline
	3. Familiarize with the required technologies & 100 & 25 & 25 & 25 & 25 & n/a \\ \hline
	4. Requirement Document & 90 & - & 100 & - & - & Document will be revised for later milestones \\ \hline
	5. Design Document & 100 & 33 & 33 & - & 33 & n/a \\ \hline
	6. Test Plan & 50 & - & 10 & 40 & - & Test plan will be revised for milestone 2 \\ \hline
	7. Progress Evaluation & 100 & 10 & 70 & 10 & 10 & n/a \\ \hline
	8. GIT test push & 100 & 25 & 25 & 25 & 25 & n/a \\ \hline
	9. Python hello world & 100 & 25 & 25 & 25 & 25 & n/a \\ \hline
	\end{tabularx}

	\subsection{Discussion of each task}
	\subsubsection{Select Collaboration Tools}
	This task included of setting up the following collaboration tools:
	\begin{enumerate}
		\item GIT repository
		\item Webpage
		\item Slack team communication
		\item Slack shared calendar
		\item Slack shared and personal to do list
		\item Google slides
		\item Google docs
	\end{enumerate}
	\subsubsection{Investigate packages, languages and tools}
	The following data mining tools were explored and pros and cons listed: \\ 
	
	\begin{tabularx}{\textwidth}{|X|X|X|X|}
	\hline
	\textbf{Technology} & \textbf{Notes} & \textbf{Pros} & \textbf{Cons} \\ \hline
	Advanced Miner & see \ref{adv}&see \ref{adv}& see \ref{adv}\\ \hline
	Ghost Miner & see \ref{ghstm} &  see \ref{ghstm}& see \ref{ghstm} \\ \hline
	GNOME & see \ref{app_shi} & see \ref{app_shi} & see \ref{app_shi}  \\ \hline
	SPAD & see \ref{app_shi} & see \ref{app_shi} & see \ref{app_shi}  \\ \hline
	ESTARD & see \ref{est} & see \ref{est} & see \ref{est}\\ \hline
	Statistica & Not freeware, Changed to WEKA (see \ref{app_mo}) & see \ref{wek} & see \ref{wek} \\ \hline
	Tanagra & see \ref{app_yaq} & see \ref{app_yaq} & see \ref{app_yaq} \\ \hline
	Rapid Miner & see \ref{app_yaq} & see \ref{app_yaq} & see \ref{app_yaq} \\ \hline
	\end{tabularx}

	\subsubsection{Familiarize with required technologies}
	This task included of familiarizing with GIT, google docs, Python and 8 of the assigned tools. Hello world like programs were pushed on the repository for python and git.
	\subsubsection{Requirements, Progress Evaluation and Design Documents, Test Plan}
	The documents were created according to the guidelines provided on http://cs.fit.edu/~pkc/classes/seniorProjects/
	\subsubsection{GIT test push and Python Hello World}
	The corresponding files were pushed on to the repository.
	\subsection{Discussion of team member contribution}
	\subsubsection{Shreyas}
	Created GIT repository and project website and set up Slack team communication. Researched and provided brief report on AdvancedMiner and GhostMiner. Authored the progress evaluation and the Requirements document contributed to Design and Test Plan document and presentation.
	\subsubsection{Shiru}
	Set up calendar and todo list plugin for slack. Researched and provided report on GNOME and SPAD. Contributed to all 4 documents and presentation. Explored new technologies including GIT. 
	\subsubsection{Yaqeen}
	Provided detailed report on Tanagra, SPAD and Rapid Miner. Set up google slides. Contributed to all 4 documents and presentation.
	\subsubsection{Mohammed}
	Provided report on ESTARD and statistica. Set up google docs. Contributed to all 4 documents and presentation.
	
	\section{Plan for the next milestone}
	\begin{tabularx}{\linewidth}{|X|X|X|X|X|}
	\hline
		\textbf{Task} & \textbf{Shreyas} & \textbf{Yaqeen} & \textbf{Shiru} & \textbf{Mohammed} \\
		\hline
		Finalize conceptualizing 4 behaviors & & & & \\ \hline
		Finalize packages, languages and tools & & & & \\ \hline
		Implement and test framework to get input from syllabus & & & & \\ \hline
		Select data mining techniques for each behavior & & & & \\ \hline
		Provide report explaining behaviours and corresponding data mining methods & & & & \\ \hline
		Implement and extract data for 2 behaviors & & & & \\ \hline
		Update Requirement document & & & & \\ 
		\hline
	\end{tabularx}

	\subsection{Discussion of each task}
	\subsubsection{Finalize conceptualizing 4 behaviors } \label{behs}
	Conceptualizing a behavior will include examining the data and thinking of common behaviors that can be modelled from the data. There is no programming involved here, more of common sense and process analysis.
	\subsubsection{Finalize packages, languages and tools}
	Select between Java, Python or suitable frameworks to integrate both. Research 1 more tool per person and finalize tools for each behavior. 
	\subsubsection{Implement and test framework to get input from syllabus}
	A program that will read through the provided syllabus files in order to extract data like due dates, test dates etc.
	\subsubsection{Select data mining techniques for each behavior} \label{techs}
	Select a data mining technique for each of the behaviors explored in \ref{behs}. Specify language, packages and tools.
	\subsubsection{Provide report explaining behaviours and corresponding data mining methods}
	Provide report of work from \ref{techs} and \ref{behs}
	\subsubsection{Implement and extract data for 2 behaviors}
	Implement atleast two behaviour extractions.
	\subsubsection{Update Requirement document}
	Update requirements document with all the specific data mining techniques that have been finalized in \ref{techs}
	\pagebreak
	\section{Sponsor feedback on each task for current milestone}
	\pagebreak
	\section{Appendices}
	\subsection{Shreyas' Tool research report} \label{app_shr}
	\subsubsection{Advanced Miner} \label{adv}
	Advanced miner seems to be an all in one tool that has features ranging from preparing data to creating specialized reports like risk analysis. This tool is a great resource for statistical analysis, but lacks the more lower level commands. \\
	The language support includes java hadoop and SQL which are result of its big data specialization. The tool is also capable of creating formatted reports.\\
	This may be a good fit for statistical aspects of the product but the community version has limited functionality in terms of memory usage thus restricting the scope which the free version can be used for. This tool may be effective in order to attain abstractions for some math pivoted data mining methods.
	\subsubsection{Ghost Miner} \label{ghstm}
	Developed by fujitsu Ghost Miner is a great tool that supports databases like spreadsheets as well as complex machine lerning algorithms. \\
	The provided packages are highly flexible in terms of different data. Ghost miner also provides visualization support. Ghost miner has an intuitive GUI that facilitates ease of use.
	\subsection{Shiru's Tool research report} \label{app_shi}
	Gnome data-mine tools is growing collection of tools packaged to provide a freely available single collection of data mining tools. Sometimes, gone data-mine includes underlying command-line data mining applications and some other GUIs for these applications.	Gnome-data-mine-tools works on the Linux platform, and support C/Python language. Gnome-data-mine-tools is free download online. It is work on Linux platform. After we installed it. There are includes some tools that in the package. For example, Apriori Association Rules, Bayes Classifier, Decision trees. GDM Apriori Association Rules is a Gnome2 GUI for building association rules from transaction data. GDM Bayes is a Gnome2 GUI for building Bayes classifiers from data. GDM Decision Tree is a Gnome2 GUI for building decision trees from data.
	Actually, I don’t think GNOME is a good way data mining for our project. Because there have some problems for this tool. The first one is GDM does not provide API for program to extend function, it means programmers need to read and extract the source code by themselves. It is inconvenient. The second reason is GDM only work on Linux platform, cannot work on other platforms. I think one tools that support multiple platforms is better.
	\subsection{Mohammed's tool research report} \label{app_mo}
	\subsubsection{ESTARD} \label{est}
	ESTARD Data Miner is an easy to use tool that is designed for both skilled and regular users. One of the main strengths of ESTARD is predictive analysis which would be very helpful in this project, because predicting the behaviors that lead to a better grade is one of the main tasks in this project. Another strength of ESTARD is that it is able to find hidden relations in both structured and unstructured sets of data. ESTARD has many useful features that provide different types of output, some of these features include What-if statistics, decision rules, and decision trees.  There are not any significant disadvantages of using ESTARD for this project. However, more tools are necessary for better and more accurate information extracted in the project.
	\subsubsection{WEKA} \label{wek}
	WEKA contains machine learning algorithms written in Java for data mining tasks. This tool is fairly easy to use; it can perform many data mining tasks such as data preprocessing, visualization, clustering, regression and feature selection. Weka is widely used in several different programs such as algorithms for data analysis and predictive modeling. One disadvantage of Weka is that it does not support multi-relations data mining. Overall, this tool would be a good addition to our data mining tools selection as it is rated one of the top tools for finding data mining. 
	\subsection{Yaqeen's tool research report} \label{app_yaq}
	Finding ways to extract the information that we need from the data is one of the important steps toward completing the goal of this project. The two tools that I have been exploring are Tanagra and SPAD. \\
\hfill \break
SPAD is a mature data mining suite that provides powerful exploratory analyses and data mining tools, including PCA, MCA, clustering, interactive decision trees, discriminant analyses, text mining and more, all via user-friendly GUI. SPAD data mining approaches are Classification Discovery, Cluster Discovery, Regression Discovery, Association Discovery, Text Mining, Outlier Discovery, Data Visualization, Discovery Visualization, Social Network Analysis. Although SPAD is a powerful data mining software, it is a commercial licensed software that comes in French. These two points in SPAD will delay the process of extracting information. Also, the only website that we can download SPAD from is www.coheris.com, which is in French, too. In my opinion, SPAD is not the data mining software that we want to be using in this project. \\
\hfil \break
Tanagra, data mining suite, is a free software for academic and research purposes. It proposes several data mining methods from exploratory data analysis, statistical learning, machine learning and databases area. Tanagra is the successor of SIPINA which implements various supervised learning algorithms, especially an interactive and visual construction of decision trees. TANAGRA is more powerful, it contains some supervised learning but also other paradigms such as clustering, factorial analysis, parametric and nonparametric statistics, association rule, feature selection and construction algorithms. Moreover, it’s an "open source project" as every researcher can access to the source code, and add his/her own algorithms. Also, Tanagra’s main purpose is to give researchers and students an easy-to-use data mining software, conforming to the present norms of the software development in this domain (especially in the design of its GUI and the way to use it), and allowing to analyze either real or synthetic data. In the other hand, TANAGRA does not include what makes all the strength of the commercial software in this domain; a wide set of data sources, direct access to datawarehouses and databases, data cleansing, interactive utilization ..etc. The last thing, Tanagra originally comes in French as well, but we can find it in English, too. From this concept, I think using Tanagra is better than using SPAD, but I wouldn’t recommend it. \\
\hfil \break
While exploring SPAD and Tanagra, I researched rapid miner. Weka, a software that I think is helpful in data mining, which is originally in JAVA. Weka is a collection of machine learning algorithms for data mining tasks. The algorithms can either be applied directly to a dataset or called from the researcher own Java code. Weka contains tools for data pre-processing, classification, regression, clustering, association rules, and visualization. It is also well-suited for developing new machine learning schemes. Weka is open source software issued under the GNU (General Public License). Also, one of the helpful things about Weka is that you can apply it to big data. \\
\hfill \break
In general, all the data mining softwares above are useful, but in order to choose a software that is most helpful we have to consider the main goal of the project. In my opining, from all the three data mining softwares, Weka is what I think most helpful. SPAD is powerful commercial tool that is not free nor is in English, which makes it hard to work with. Also, Tanagra doesn’t include all what a commercial software would offer which makes it less useful in our case. In conclusion, Weka what I see best from what I gathered from my research. 

	\pagebreak
	\section{Sponsor Evaluation}
	Sponsor: Please detach this page and return to Dr. Shoaff \\ \hfill \break 
	Score (0-10) for each member: circle a score (or circle two adjacent scores for .25 or write down a real number between 0 and 10) \\ \hfill \break
	\begin{tabularx}{\textwidth}{|X|c|c|c|c|c|c|c|c|c|c|c|c|c|c|c|}
	\hline
	Shreyas Ugemuge & 0 & 1 &  2 & 3 & 4 & 5 & 6 & 6.5 & 7 & 7.5 & 8 & 8.5 & 9 & 9.5 & 10 \\ \hline
	Yaqeen AlKathiri & 0 & 1 &  2 & 3 & 4 & 5 & 6 & 6.5 & 7 & 7.5 & 8 & 8.5 & 9 & 9.5 & 10 \\ \hline
	Mohammed AlHabsi & 0 & 1 &  2 & 3 & 4 & 5 & 6 & 6.5 & 7 & 7.5 & 8 & 8.5 & 9 & 9.5 & 10 \\ \hline
	Shiru Hou & 0 & 1 &  2 & 3 & 4 & 5 & 6 & 6.5 & 7 & 7.5 & 8 & 8.5 & 9 & 9.5 & 10 \\ 
	\hline 
	\end{tabularx}
	\hfil \break
	\hfil \break
	\namesigdate{Faculty Sponsor}
	\end{document}